% 001000 IDENTIFICATION DIVISION.
\documentclass[11pt]{article}


% 002000 ENVIRONMENT DIVISION.
\usepackage[T1]{fontenc}
\usepackage{ae,aecompl}
\usepackage[margin=1in]{geometry}
\usepackage{microtype}
\usepackage{hyperref}
\usepackage{ifthen}
\usepackage{parskip}

\newcommand\question[2]{\item{\it #2}\vskip .5em}
\newcommand\code[1]{\mbox{\texttt{#1}}}


% 003000 DATA DIVISION.
\title{CSCI 455 Project 1 Discussion}
\author{Your Name}


% 004000 PROCEDURE DIVISION.
\begin{document}


\maketitle


\begin{enumerate}

\question{3}{
\label{q:init-criterion}
Which criterion of the critical section problem is not satisfied by the initial
implementation?  Explain.
}

% Answer here.

\question{5}{
Describe the cats-and-dogs synchronization problem in terms of the events that
each pet must wait for.
}

% Answer here.

\question{5}{
Do you need to add another synchronization construct to the implementation to
handle birds?  Explain.
}

% etc...

\question{5}{
Why would using a broadcast instead of signaling pets one at a time result in
more useless wakeups than are necessary?
}

\question{7}{\label{q:last-proj}
Meeting the \textit{bounded waiting} criterion for the critical-section
problem was not required for this assignment, but it is possible that your
solution does.  If your implementation satisfies it, explain why.  Otherwise,
describe what additional variables or constructs that would be needed to
implement it and when they would be updated.
}

\question{5}{
The rules regarding the wolves provide an example of a classic synchronization
problem (one of the ones mentioned in class).  Which problem is it?  What do the
wolves correspond to and why?
}

\question{5}{
This assignment initially used the term ``deadlock'' to refer to the situation
where a pet is never woken up, even after the room empties.  Why specifically
was this the wrong term to use?
}

\question{10}{
Suppose the pet shop wants to give their pets more time to play.  They buy the
office next door and create a separate shop identical to the first, with the
same rules.  They want to be sure that both playrooms are being used as much
as possible, so if the pets in Shop A are not currently playing, some pets
will be brought from Shop B and vice versa.  Describe an analogous issue from
the textbook's chapter on scheduling and how it would apply to the pet shop's
logistics in this scenario.
}

\question{5}{\label{q:last-hw}
If the two shops were connected instead, so that each pet could choose to wait
for Playroom A or Playroom B, would it be possible to reach deadlock?
Explain.
}

\end{enumerate}

\end{document}
